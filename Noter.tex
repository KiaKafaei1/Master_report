\section{Noter}
\subsection{Noter til Paper}
Trying to solve “Accurate and efficient indoor path planning” since a lot of people uses indoor walking. There are challenges with indoor navigation.
\\\\
Industry Foundation Classes (IFC)
Semantic information in BIM models
Revit
Triangular prism subdivision
Geometric and semantic information
2D grid subdivision and 3D space subdivision based on triangular prism.
How about if you use variable sizes of grid, depending on the rooms.


\subsection{Noter til master these:Pathfinding in Two-dimensional
Worlds }

You can represent the 2D world in 2 ways, as a 8-grid system and a polygonal way. The problems with the grid system are suboptimal path length and the problem of choosing a fitting grid resolution.

Polygons works by connecting each portruding corner and making a connection graph.

A* works on graphs and grids
JPS (jump point search) works on grids only, and is good for large areas of open space.

HPA is also a good algorithm for grid systems but is not necessarily optimal. At most 1\% worse than optimal. Makes use of preprocessing, and clusters the map and makes a graph of higher abstractions such that it gets an overview of the overall structure of the map before doing the pathfinding.
A fitting cluster size is also of importance for HPA* as we saw in the Scaled Maze map.

VG (visibility graph)
is an algorithm that works on polygonal maps.



\subsection{Noter process og generelle tanker}
\subsubsection{Rapport tanker}
Rapporten består af undersøgende arbejde og er vigtigere end implementationen.
Rapport bliver vurderet på diskussion.
Man kan gøre det på mange måder.
Ulempen ved connectivitet er at man ikke kan se væggene. Graph måden.
Find så mange metoder som muligt og skriv hvorfor de andre er dårlige. 
Det viser at du har tænkt dig. 
Rapport stuff: for og imod forskellige implementationer.
Kriterie liste med fordele og ulemper.
Undersøg begge veje, og argumenter.
Undersøg på papir og læs dig frem til det. Skriv for og imod for begge. 
Læs en masse artikler, og skriv alle tanker omkring det.

\subsubsection{Ugentlige rapporter}
Hvad skete I den forgangne uge, hvad har været nogle problemer.
Ugerapport der beskriver status.
Skriv nogle noter til jer selv.
Små tekst stumper
Trello planlægning, to do liste
Include figures and bibtex and notes.


\subsubsection{Brainstorm}
Lav en meget præcis liste med ting du skal have gjort så du altid har noget at lave.
Brug en time på det i to doist.

Hvad skal jeg gøre i sidste ende?
Robotten skal kunne bevæge sig fra punkt til punkt i en bygning mens den tager billeder på en etage. Den skal have en strategi til hvordan den går rundt og hvordan den undviger forhindringer.

En meget overordnet planlægning af hvordan du kommer systematisk rundt i en bygning, så den kan blive scannet.
Denne opgave behøver ikke være så geometrisk. Du kan egentlig se en bygning som en graf, hvor hver knude svarer til et rum.

Det andet element er at komme rundt i de enkelte rum og gå fra rum til rum på en hensigtsmæssig måde.
How to compare different methods?

\subsection{Thesis noter}
Explain the fundamental problems which will be the background of your theses.
Maybe build it up as a story.
Say not what you are going to do but why.
Start with the why. Elevator pitch in the first section.
Remember that you are writing with human beings. Have a good flow. Have 1.5 months for report writing.
Nudge people into the topic and not straight into it.

Related works doesn't have to be detailed.
Theory of the things you use is background
Present papers in related work, provide an overview of the theories briefly.
Background is only theory on what you have implemented.
Related might be after background.
More than 5 references, have as many as possible.
How long the report, write more background if bad contribution, more report.
The reader should not have a burning question.
Be very explicit about the scope of the report.What will you not go into.

Method is about what you have done yourself, no matter what you have followed, make proper citation 
You can also have a little introduction paragraph, saying you are following this method.
Have implementation section, the tools, the computer used.

How to show the results?
How to make results?
What results could you present that support that you finished the project.
Google sheets

All the timer consider getting credit for what you did
Get more credit by discussing the way you got there. Some of the considerations you made and the issues you solved.
Explain how you ended up with the final product and why?
By extending the method section.

We can get a quick readthrough of the report but not a detailed read


\subsection{Uge 7}
Figure out exactly whats going on with data.
It seems that it is mixing up axis.
Switching x,y for the doors
RAC look into the building 
Study the format and file
send code and description to andreas, also useful for report.

What I want to do is to remove all doors that are not connected to a wall. To do this I am going to find the distance for each point to all the walls (might change this to closest walls). Since I only have the coordinates and not the lines what I am going to do is to make a triangle including the two points of the line and the third point being the door node. I am going to find the length of each side of the triangle. Using Herons formula I will find the area, and knowing the area I will find the height of the triangle. The only issue is that the distance from the point to the line is not always equal the height of the triangle. The distance could be longer - not shorter though.

Sagen er den at RAC bygningen er ikke en bygning der eksisterer i virkeligheden (ihvertfald ikke I Danmark). Jeg har fundet 3D modellen af bygningen, hvor vægge dog ikke er inkluderet. Jeg tænker at få simuleringen til at fungere på en simpel udgave af bygningen og derefter arbejde på noget mere komplekst pathfinding. Først vil jeg dog plotte dørene.

\subsection{Uge 8}
Remove the doors that are floating, make the synthetic data clean
Find a way to place corners, start with manually placing corners, find an algorithm later in the project.
Go back later on, generally you should have a working implementation before fiddling with the  different things.
One way to implement the radius of the robot is to make all the walls thicker and keep the point as is.

\subsection{uge 9}
Not likely scenario 
Discuss with Dalux advisor 
Thicken walls with radius
Distance fields with radius
Distance fields, compute distance to closest wall
sample distance field and check for radius
Discretize the entire scene, for every point calculate distance to closest wall
Image of distance field which we can sample
spent time writing for thesis the obstacles in a more general, high level problems
find function for point line segment distance 
easy to debug

\subsection{10}
Diskretisere selve ruten
Find nærmeste væg 
KDtrees datastruktur 
litteratur på datastrukturer
case med ustabilitet
lav test setup
Diskretisere rummet med kasser
raytracing på punktet
Tradeoff mellem skanningspunkter og afdækning
tænk over corner nodes
Behæver ikke tænke på loxalization lige nu
Tag hensyn til søjler når du scanner 
Tænk over kasser i rummet problemet

En optimal strategi kunne være at placere midt i rummet og tage billeder dér.

Find en måde at autogenere rum noder som er placeret midt i rummet

Læs op på slam og hvordan det fungerer.

Find en måde at autogenerere corner nodes.

Behøver robotten localization hvis vi giver den en map?

Lave autogenerering af ønskede noder.

Pros and cons: 
Distance field:
- Fordelen ved at lave et diskret distance field er, at du kan udregne det væg for væg, og når du er færdig, så kan du bruge det - og det er lynhurtigt at slå op i et billede. 

Diskret rute:
Pros:
- Bemærk dog at du ikke behøver at beregne afstanden med fast interval i alle tilfælde. Hvis nu robottens radius er 0.5 m og der er 10 m til nærmeste væg, så kan robotten jo bevæge sig 9.5 m før den rammer noget. Det kan gøre det hele lidt hurtigere.
- Jeg ville mene at det letteste var bare at have en masse punkter og så bruge afstandsfunktionen til at teste om robotten kan komme fra et punkt til et andet punkt ved at gå i lige linje.
Cons:
- Hvis du laver en funktion, der finder afstanden, så skal du enten løbe alle vægge igennem, hver gang du ønsker at finde afstanden - eller du skal have en datastruktur, der gør at du hurtigt kan finde den nærmeste væg. Det første er let nok men kan blive tungt hvis du har virkeligt mange vægge, og det andet tilføjer lidt kompleksitet.


I think I can drop the idea of corner nodes.

The idea in general is to implement the route in such a way that it first check if route is traversable if not it moves 90 degrees in either direction. It checks again and goes back and moves 90 degrees in the other direction and keeps checking until it finds a traversable path. It then finds the distance to the nearest wall (minus the radius) and walks that amount in a straight line towards the end node. It then moves a bit 90 degrees and sees if the distance to the nearest wall is decreasing or increasing.

It might be easiest to do discretization and do A* algorithm.


Can I avoid corner nodes?
Place the corner nodes a certain distance from the corner, equal to the radius.

We have to save the vector if we want to turn 90 degrees.
The vector is just the difference in y coordinates and the difference in x coordinates.
The amount we go left and right is the amount of distance to closest wall.

Does this way of thinking work?
I think it works in 99 \% of scenarios but we will have to think of scenarios where this doesn't work.

What do we need to make this implementation work?

There are 2 types of obstacles, the ones we know before hand and that we can plan for.
And then there are the ones that we don't know about random chair in the room etc.
For now we are only solving for the obstacles that we do know about.
Maybe it is okay that it only works 99\% of the time.

Another way is to make it keep going right until it finds a direct path to the end node.
It should only do this when it has tried the first algorithm and that doesn't work.

There are no scenarios where the combination of these two algorithms don't work.
Given the assumption that there is a direct (not straight/linear) path between the 2 nodes.

How to make the discrete route 100\% correct.

\subsection{11}
It is an active research area
Quite a bit of litteratur 
Read  the papers that Jeppe sent
One is about beloney tiangulation where it triangulates the entire area
Switching algorithm to A star
Keep only the second algortihm and remove the first algorithm
Local geomtry investigation
Keep the discussion for the report
Keep A*
This is not a navigable part, split and go in both directions
Split the robot in 2 and check in paralellel. You will get a tree.
Stopping criteria  will be a certain distance before giving up
Keep the distance field in the report
All the features should be discussed. 


